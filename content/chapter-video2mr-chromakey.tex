% !TEX root = ../thesis-example.tex
%
\section{Chroma Key}
\label{sec:chromakey}

Beginning from the camera, the video signal travels through the Inogeni 
converter and is accessible with the system API for webcams. 
(See figure \ref{fig:system-components})

The initial step is to remove the green background from the image, which should 
be greenscreen. For a reference green, there has to be a color picked 
manually in the material editor of Unity - this was made easy by a checkbox to 
show raw output from the camera. Then a middle-ground green can be 
picked. This is an important setup step, since lightning situations can vary 
greatly and minor differences in light setups can have a great effect on the 
outcome of visible green background captured by the camera.
\newline
An extreme example case is used for comparing these chroma keying variants:

\begin{figure}[htb]
	\includegraphics[width=\textwidth]{_raw_resources/Comparison_Example.png}
	\caption{Comparison Image \cite{vimeo:shia:2015} - sRGB Output}
	\label{fig:chroma:color}
\end{figure}

\subsection{Initial Assumption}

Each RGB color can be represented as a discrete 3-Vector of (red, green, blue) 
values in range of [0, 1]. The composition between two colors can be summarized 
as equation as following, where a foreground image $F$ and a background 
image $B$ - $\alpha_B$ is assumed to be $1$:

\eq{eq:chroma:assumption:alpha:1}{
	I(x, y) = \alpha(x,y) F(x, y) + (1 - \alpha(x, y)) B(x, y)
}

This matting equation has to be generalized, where $\protect\alpha$ is a value 
between [0, 1] on fore- and background, yielding a total Alpha of $\alpha_T$ as 
following:

\eq{eq:chroma:assumption:alpha:weak}{
	\alpha_T = \alpha_B * (1 - \alpha_F) \\
}

\eq{eq:chroma:assumption:alpha:cont}{
	I(x, y) = (1 - \alpha_T)F(x, y) + \alpha_T B(x, y)
}

\subsection{Euclidean RGB Difference}
Assuming a source pixel color $C_S$ and a reference color $C_R$ we can 
calculate the euclidean distance between these colors.
\eq{eq:euclidianrgb}{
	\alpha = \sqrt{(C_RR - C_SR)^2 + (C_RG - C_SG)^2 + (C_RB - C_SB)^2}
}

This is a computationally very low cost and works well enough for tell a 
difference between two separate colors. It fails to accommodate for colors that 
are perceived as different, but are tinted by the reference colors. Since the 
greenscreen will never achieve 0\% reflectivity, some residue of the background 
color will mix with the filmed actors.

\begin{figure}[htb]
	\includegraphics[width=\textwidth]{_raw_resources/Comparison_RGB_color.png}
	\caption{Chroma Keying by using euclidean RGB distance}
	\label{fig:chroma:euclidean:rgb}
\end{figure}

\subsection{Euclidean YCgCo Difference}

YCgCo stands for Luminance (Y), chrominance green (Cg) and chrominance orange 
(Co) and helps decorrelating color spaces. Since it is a fast, lossless color 
transformation it is used in example for H.264 video encoding. The two 
chrominance channels are then split into green to magenta and orange to blue 
color values and allow for a more accurate distance calculation between two 
colors.
\newline
Transforming any arbitrary RGB color to YCgCo can done with a single matrix 
multiplication:

\eq{eq:ycgco:transformation}{
	\begin{bmatrix}
		Y \\
		Cg \\
		Co \\
	\end{bmatrix}
	=
	\begin{bmatrix}
		 \frac{1}{4} && \frac{1}{2} &&  \frac{1}{4} \\
		-\frac{1}{4} && \frac{1}{2} && -\frac{1}{4} \\
		 \frac{1}{2} && 0           && -\frac{1}{2}
	\end{bmatrix}
	*
	\begin{bmatrix}
		R \\
		G \\
		B 
	\end{bmatrix}
}

Given two colors, one from the video source $C_S$ and a reference color $C_R$ 
it is now possible to calculate the euclidean distance on the two chrominance 
channels:

\eq{eq:ycgco:euclidean}{
	\alpha = \sqrt{(C_RCg - C_SCg)^2 + (C_RCo - C_SCo)^2}
}

Since the increased decorrelation, the result is more accurate and shows less 
artifacting on unwanted pixels.

\begin{figure}[htb]
	\includegraphics[width=\textwidth]{_raw_resources/Comparison_YCgCo_color.png}
	\caption{Chroma Keying by using euclidean YCgCo distance}
	\label{fig:chroma:euclidean:ycgco}
\end{figure}

\subsection{Euclidean Lab Difference}
The International Color Consortium (ICC) defined 1976 \textit{Lab $\Delta$E} as 
a standard way of calculating color differences with \textit{Lab} colors. The 
final distance calculation is the linear euclidean distance as with all other 
models, but accommodates for perceived color differences.

\todo[inline]{this is very rough and only contains equations currently used}

sRGB conversion to linear RGB in respect of energy per channel:

\eq{chroma:inversecompanding}{
	v \in \{r, g, b\} \land V \in \{R, G, B\}
}

where: 

\eq{chroma:inverseRGBcompanding}{
	v = \\
	\begin{cases}
		V / 12.92                   & \quad \text{if } V \leq 0.0405 \\
		((V + 0.055) / 1.055)^{2.4} & \quad \text{otherwise}
	\end{cases}
}

from there l

\eq{chroma:conversioncalc}{
	\begin{bmatrix}
		X\\
		Y\\
		Z\\
	\end{bmatrix}
		=
	\begin{bmatrix}
		M
	\end{bmatrix}
	\begin{bmatrix}
		R\\
		G\\
		B\\
	\end{bmatrix}
}

where:

\eq{chroma:rgbmatrix}{
	\begin{bmatrix}
		M
	\end{bmatrix}
		=
	\begin{bmatrix}
		R X_r && G X_g && B X_b \\
		R Y_r && G Y_g && B Y_b \\
		R Z_r && G Z_g && B Z_b
	\end{bmatrix}
}


\eq{chroma:xyzmatrix}{
		\begin{bmatrix}
		X\\
		Y\\
		Z\\
	\end{bmatrix}
	\begin{bmatrix}
		M
	\end{bmatrix}
	=
	\begin{pmatrix}
		X_r / Y_r && X_g / Y_g && X_b / Y_g \\
		1 && 1 && 1 \\
		\frac{1-X_r-Y_r}{Y_r} && \frac{1-X_g-Y_g}{Y_g} && \frac{1-X_b-Y_b}{Y_b}
	\end{pmatrix}
}

Where $\begin{bmatrix}M\end{bmatrix}$ for RGB D65 is:

\eq{chroma:sRGBD65}{
	\begin{bmatrix}
		0.4124564 && 0.3575761 && 0.1804375 \\
		0.2126729 && 0.7151522 && 0.0721750 \\
		0.0193339 && 0.1191920 && 0.9503041 \\
	\end{bmatrix}
}

Based on a reference white $U_r \in \{X_r, Y_r, Z_r\}$:

\eq{chroma:xyz2lab:defines}{
	U \in \{X, Y, Z\} \land W \in \{L, a, b\}
}

\eq{chroma:xyz2lab:epsilon}{
	\epsilon = 0.008856 \land \kappa = 903.3
}

where:

\eq{chroma:xyz2lab:ref}{
	w_r = \frac{U}{U_r}
}

\eq{chroma:xyz2lab:channel}{
	f(w) = \\
	\begin{cases}
		\sqrt[3]{w_r}      & \quad \text{if } U > \epsilon \\
		\frac{\kappa w_r + 16}{116} & \quad \text{otherwise}
	\end{cases}
}


\eq{chroma:xyz2lab:conversion}{
	\begin{bmatrix}
		L \\
		a \\
		b \\
	\end{bmatrix}
	=
	\begin{bmatrix}
		116 f_y - 16 \\
		500 (f_x - f_y) \\
		200 (f_y - f_z)
	\end{bmatrix}
}

With this conversion from sRGB to linear RGB to XYZ to Lab we can now calculate 
the euclidian linear distance between two colors $C_1$ and $C_2$, which already 
have been converted to Lab:

\eq{chroma:cie76:distance}{
	\Delta E = \sqrt{(C_2L - C_1L)^2 + (C_2a - C_1a)^2 + (C_2b - C_1b)^2}
}

These values are rated by their perceptive difference \cite{mokrzycki:2012}:

\begin{tabular}{l | l}
	0.0 \dots 0.5 & the difference is unnoticeable \\
	0.5 \dots 1.0 & the difference is only noticed by an experienced observer \\
	1.0 \dots 2.0 & the difference is also noticed by an unexperienced observer 
	\\
	2.0 \dots 4.0 & the difference is clearly noticeable \\
	4.0 \dots 5.0 & fundamental color difference  \\
	> 5.0 		  & gives the impression that these are two different 
	colors        
\end{tabular}

Now it's possible to map alpha values for each pixel based on $\Delta$E 
distances between $m, n$ by clamping and biasing $\Delta$E:

\eq{chroma:cie76:clamp}{
	f({\Delta E}) = x = \frac{\Delta E - n}{m - n}
}

\eq{chroma:cie76:min/max}{
	\alpha_{\Delta E} =
	\begin{cases}
		n        & \quad \text{if } x \leq n \\
		x        & \quad \text{if } n \leq x \leq m \\
		m        & \quad \text{if } m \leq x
	\end{cases}
}

\eq{chroma:cie76:mapping}{
	\alpha(I(x, y)) = 3\Delta E^2 - 2\Delta E^3
}

\begin{figure}[htb]
	\includegraphics[width=\textwidth]{_raw_resources/Comparison_DeltaE_color.png}
	\caption{Chroma Keying by using $\protect\Delta $E distance}
	\label{fig:chroma:euclidean:rgb}
\end{figure}

