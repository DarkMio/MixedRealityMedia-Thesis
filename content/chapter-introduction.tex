% !TEX root = ../thesis-example.tex
%
\chapter{Introduction}
\label{sec:intro}

\cleanchapterquote{If a technological feat is possible, man will do it. Almost 
as if it's wired into the core of our being.}{Motoko Kusanagi}{(Ghost in the 
Shell)}

Extending reality with the help of computer generated imagery is no new 
concept. Ever since real time 3D graphics was possible there was an attempt to 
extend the understanding of reality. Within the recent years there have been 
great successes in the industry, most notably in image augmentation was 
"Pok\'emon Go" with an estimated install base of 750 million downloads 
worldwide in June, 2017. \cite{appannie:2017}  Just before this thesis 
started, Apple and Google showed off their consumer-ready hard- and software 
for augmented reality experiences.
\newline
Virtual Reality Head Mounted Displays have had a similar push in sales with an 
approximate of 5.83 million sold devices, which range in a sales price between 
80 - 900€ for a VR kit, ranging from the very simple Google Daydream View and 
the very sophisticated HTC Vive. \cite{erguerel:2017} And in these figures are 
the sales of Google Cardboards missing, which is approximated at around 80 
Million.
\newline
This generation of computer systems, in which are PC workstations, game 
consoles and smartphones, is finally sophisticated enough in computation speed 
and sensor-sensitivity to allow low latency tracking, precise to just a few 
millimeters.

\section{Overview}
\label{sec:intro:outline}

The idea of Virtual Reality (VR) and Head Mounted Displays (HMDs) stems from a 
cultural need to switch into roles of foreign worlds. Through the advancing 
development of hard- and software over the last decades emerges a medium which 
has unmatched immersion and creates an unique, transforming experience into any 
imaginable environment.

VR and HMDs are now advanced enough for consumer markets - but it stumbles at 
communicating the experience. Without having ever put on a VR-Headset it is 
nearly impossible to understand - or even imagine - what the virtual reality 
experience means. Any observer of Virtual Reality, usually done by showing what 
the VR actor is seeing, will not be able to get an understand of the importance 
and shift of reality perception without wearing the headset himself.
\newline
Showing the video output from a HMD as marketing material is contradicting with 
classic motion video productions. There is even only one famous example where 
the perspective of a First Person Shooter is reenacted, which was in the 
overwhelmingly negatively received Doom (2005) movie.

The VR industry, including but not limited to game developers, exhibition 
creators and creative studios is in need of better communication of their 
products that includes more than the current headset wearer and allows for a 
similar, adapted and immersive experience.
\newline
The currently method is called "Mixed Reality" (MR) and uses an external camera 
with the same tracking hardware of the headset to produce a video signal that 
shows the real world actor with the environment around him. There are currently 
three main ways of producing MR footage - where as only one variant allows for 
live compositing with highly accurate imaging results.


\section{Motivation}
\label{sec:intro:motivation}

My early teenage years started around the time where digitalization and global
interconnectivity begun and broadband Internet became commercially available.
Suddenly remote multiplayer games, unlimited image sharing - and yes, music
sharing, too -, Java-Applets, Flash, HTML framesets and "Marquee" CSS emerged in
that medium. 3D Acceleration became a de-facto standard and even simple office
PCs got weak, but dedicated graphics processing units built in. The mass of
pixels by increasing the resolution of displays was basically a yearly
iteration in greater, better, smaller and brighter.

I am personally very interested and invested in Virtual/Mixed/Augmented Reality 
to succeed and liked the idea to merge multiple forms of media into one - which 
is, in my personal opinion, a great summary of my studies and its contents. 
This thesis represents my interests and the reasons why I chose these studies.

\section{Problem Statement}

Initially I will research motion video productions, computer generated imagery 
and color theory. This leads to the knowledge to implement basic, interactive 
live motion video.
\newline
The core aspect will be integrating a multitude of Hardware in a software that 
allows for dynamic video compositing in 3D environments at runtime while a user 
is interacting with the virtual reality scene. This allows that the person 
using the Vive HMD to be composited into the scenery and it looks like he is in 
that scene standing. The essential difference between classic post production 
is, that this system is planned to operate on runtime, allow additional 
observers to get an interesting composited imagery of what the VR actor is 
experiencing.
\newline
An additional extension is to dynamically track the camer position, allowing 
for dynamic camera movement and a freely moving actor. 
\section{Relevance \& Challenges}
\label{sec:intro:relevance}

Probably should be called Relevance \& Scope.

\section{Results}
\label{sec:intro:results}

Result stuff

\section{Thesis Structure}
\label{sec:intro:structure}

This thesis gets contemplated by digital, mostly motion video, material hosted 
on GitHub. Print is a great medium, but lacks the ability for short 
demonstrations of video imaging solutions, problems and edge cases. To 
visualize these problems properly, all video media will have an annotation for 
cross referencing on the website. It is strongly suggested to follow these 
links, they will be sorted by chapters.