% !TEX root = ../thesis-example.tex
%
\chapter{Introduction}
\label{sec:intro}

\cleanchapterquote{Top 10 Reasons why you won't believe this crazy 
quote.}{Buzzfeed}{(Nothing notable, yet always there)}

\Blindtext[2][2]

\section{Overview}
\label{sec:intro:outline}

The idea of Virtual Reality (VR) and Head Mounted Displays (HMDs) stems from a 
cultural need to switch into roles of foreign worlds. Through the advancing 
development of hard- and software over the last decades emerges a medium which 
has unmatched immersion and creates an unique, transforming experience into any 
imaginable environment.

VR and HMDs are now advanced enough for consumer markets - but it stumbles at 
communicating the experience. Without having ever put on a VR-Headset it is 
nearly impossible to understand - or even imagine - what the virtual reality 
experience means. Any observer of Virtual Reality, usually done by showing what 
the VR actor is seeing, will not be able to get an understand of the importance 
and shift of reality perception without wearing the headset himself.
\newline
Showing the video output from a HMD as marketing material is contradicting with 
classic motion video productions. There is even only one famous example where 
the perspective of a First Person Shooter is reenacted, which was in the 
overwhelmingly negatively received Doom (2005) movie.

The VR industry, including but not limited to game developers, exhibition 
creators and creative studios is in need of better communication of their 
products that includes more than the current headset wearer and allows for a 
similar, adapted and immersive experience.
\newline
The currently method is called "Mixed Reality" (MR) and uses an external camera 
with the same tracking hardware of the headset to produce a video signal that 
shows the real world actor with the environment around him. There are currently 
three main ways of producing MR footage - where as only one variant allows for 
live compositing with highly accurate imaging results.


\section{Motivation}
\label{sec:intro:motivation}

My early teenage years started around the time where digitalization and global
interconnectivity begun and broadband Internet became commercially available.
Suddenly remote multiplayer games, unlimited image sharing - and yes, music
sharing, too -, Java-Applets, Flash, HTML framesets and "Marquee" CSS emerged in
that medium. 3D Acceleration became a de-facto standard and even simple office
PCs got weak, but dedicated graphics processing units built in. The mass of
pixels by increasing the resolution of displays was basically a yearly
iteration in greater, better, smaller and brighter.


\section{Problem Statement}

Here is the actual problem in human readable words.

\section{Relevance \& Challenges}
\label{sec:intro:relevance}


\section{Results}
\label{sec:intro:results}

Result stuff

\section{Thesis Structure}
\label{sec:intro:structure}

