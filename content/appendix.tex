% !TeX spellcheck = en_US
% !TEX root = ../thesis-example.tex
%

\newglossaryentry{computer}{
	name={computer},
	description={we tricked a rock into thinking by putting lightning into it.},
	plural={computers}}

\newglossaryentry{framebuffer}{
	name={framebuffer},
	description={It usually contains ARGB data of each remaining fragment. It 
	can, however, contain any form of data, including fragment vector depth, 
	depth-normals, normals and so on.},
	plural={framebuffers}
}

\newglossaryentry{render texture}{
	name={RenderTexture},
	description={Unity exposes frame buffers as render textures. It is not 
	possible to access both dat and depth buffer on one frame buffer, which can 
	be circumvented by seperate RenderTextures with no data or depth buffer.}
}

\newglossaryentry{fragment}{
	name={Fragment},
	description={It is a rasterized pixel of a vertex (usually a triangle).}
	plural={fragments}
}

\newglossaryentry{head mounted display}{
	name={Head Mounted Display},
	description={A general term for display solutions mounting on the users 
	head. Recent examples with spatial tracking are HTC Vive, Oculus Rift and 
	Microsoft HoloLens. Untracked examples are Google Glasses or to an extend 
	Google Cardboard.},
	plural={Head Mounted Displays}
}

\newglossaryentry{6dofg}{
	name={6DOF},
	description={Describes free movement of a rigid body in 
		three-dimensional space - specifically forward/backward (surge), 
		up/down (heave), left/right (sway) translations, combined with changed 
		of rotations around all three axis. \textbf{may need wikipedia cite}
	}
}

\newglossaryentry{6dof}{
	type=\acronymtype,
	name={6DOF},
	description={Six Degrees of Freedom (6DOF) \glsadd{6dofg}},
	see=[Glossary:]{6dofg}
}

\newacronym{6DOF}{6DOF}{Six Degrees of Freedom}

\newacronym[longplural={Frames per Second}]{acr:fps}{fps}{frames per second}