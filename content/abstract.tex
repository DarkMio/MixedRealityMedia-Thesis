% !TeX spellcheck = en_US
% !TEX root = ../thesis-example.tex
%
\chapter*{Zusammenfassung}
\label{sec:abstract}
\vspace*{-10mm}

VR ist kein entfernter Traum mehr, doch die Technologie hat ein 
Marketing-Problem: Ein Blick durch die Augen des Headset-Trägers ist 
uninteressant und verliert den Nutzerkontext. Der First-Person Sicht ist nur 
schwer zu folgen und zeigt unruhige, unnatürliche Bewegungen.

Diese Bachelor Arbeit beschäftigt sich mit dem Aufbau einer allgemeinen Mixed 
Reality Pipeline für Echtzeit 3D Engines, eine Aufarbeitung von 
Echtbildaufnahmen eines VR Nutzers und die Rückführung in eine virtuelle Szene. 
Durch mehrere Videoverfahren wird das Bild des Trägers mit der virtuellen Welt 
zusammengeführt und mit den Parametern der virtuellen Welt erweitert.
\newline
Dadurch können zum Beispiel Lichtverhältnisse der virtuellen Szenerie 
nachempfunden werden und einen immersiven und einladenden Blick in den 
virtuellen Raum geboten werden.

\begin{center}
	\hrulefill
\end{center}

\vspace*{-20mm}
{\let\clearpage\relax\chapter*{Abstract}}
\vspace*{-10mm}

Virtual Reality is no distant dream anymore, but the technology has a 
marketing problem: A view through the eyes of a head-mounted display wearer 
is bland and loses the usage context. Following the first-person view is hard 
and it shows twitchy, unnatural motion.

This thesis discusses a general rendering pipeline for runtime 3D engines for 
Mixed Reality Media, a form of video composition that places a real person 
inside a virtual reality scene. The real world video will be augmented by 
multiple video techniques and parameters from the virtual environment.
\newline
This allows for example the recreation of light conditions from the virtual 
scenery and creates an immersive and inviting view into the virtual scenery.