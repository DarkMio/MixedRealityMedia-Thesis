% !TeX spellcheck = en_US
% !TEX root = ../thesis-example.tex
%
\section{Mitigating Frame Jitter}

An additional step that can be handled by the same algorithm is the mitigation 
of frame or time jittering, a term that describes a effect of different running 
framerates of an actors captured video footage and rate of 3D environment 
rendering. Since the native framerate of the HMD is higher than the frames 
produced by the video feed, there will be noticeable small jitters of virtual 
camera movement, which are not present on the source video material. The HTC 
Vive Controller are very good at picking up minuscule changes in motion, which 
are instantly translated to the transform of the camera rig and then result in 
minor motion of the 3D environment. Visually this shows in a shaking virtual 
world while the real world video stands still.
\newline
To minimize the effect, it is possible to overwrite framebuffers for as long as 
one duration of a video frame and then swapping out these buffers. The headset 
recommends to run it at 90fps - miss timings are no issue either, since it 
results in non noticeable errors in virtual projections while the displayed 
frame is stable visual performance is unaffected. Thus it is possible to write 
less than three times into a framebuffer, which is mitigated again by 
displaying the final result later - the final image is therefore unaffected 
besides imperceptible differences between the real world camera and its virtual 
position.

\todo[inline]{needs then lstlisting when the code further up is modified}
