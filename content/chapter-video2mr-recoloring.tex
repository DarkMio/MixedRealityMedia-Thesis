% !TEX root = ../thesis-example.tex
%
\section{Additional Coloring Operations}

Finally we have created the best possible recreation of a VR actor inside the 
virtual reality scene. Now we can follow up with post effects on the video feed 
to fit it to a better degree into the environment. This can be done with 
regular coloring operations, like hue rotation, brightness, contrast and 
saturation procedures on the video alone.

\subsection{Color spill removal \& Recoloring}

The green box as background spills - so to say - its green color on to the 
actor to a certain degree. With proper lightning setups it is possible to 
mitigate its effects but color retouching is always a necessity. \textbf{YCgCo} 
is, again, good enough to perform this color operation, thanks to its color 
decoupling properties. By splitting a RGB image into YCgCo $C_{Input}$ (see 
equation \eqref{eq:ycgco:transformation}) and then shifting towards the 
anti-color of the key color $C_{Key}$ for a factor weight $W \in [0, 1]$:

\eq{eq:retouch:spill:1}{
	R = \frac{C_{Key Cg, Co} \cdot C_{Input Cg, Co}}{C_{Key Cg, Co} \cdot 
	C_{Key Cg, Co}}
}

\eq{eq:retouch:spill:2}{
	\begin{bmatrix}
		Y  \\
		Cg \\
		Co \\
	\end{bmatrix}
	= 
	\begin{bmatrix}
		Y \\
		C_{Key Cg} * (R + 0.5) * W \\
		C_{Key Co} * (R + 0.5) * W
	\end{bmatrix}
}

\todo[inline]{"proper lightning setup" would be something for the appendix}

\subsection{Brightness, Contrast and Saturation}

ghi